\documentclass[a4paper, 11pt]{MyReport}

\usepackage{lscape}

\newcommand{\erd}[1]{{\bfseries \itshape #1}}

\author{Margaux BRULIARD - Nahida BENHAFFAF}
\title{DodoCiné \\ \vspace*{0.5cm}Projet Programmation Web}
\date{20 Mai 2021}

\definePrintSourceCodeStyle


\begin{document}
	\begin{printTitle}
	\end{printTitle}

	% ------------------------------------------------------------------------------
	%	TABLE OF CONTENTS
	% ------------------------------------------------------------------------------
	\tableofcontents
	\newpage

	% ------------------------------------------------------------------------------
	% 	PRESENTATION DU PROJET
	% ------------------------------------------------------------------------------
	\chapter*{Introduction}
	\addchaptertable{Introduction}
		
		\section*{Présentation du projet}
		\addsectiontable{Présentation du projet}

			\todo{écrire l'introduction }
		

		\section*{Mise en production}
		\addsectiontable{Mise en production}

			\todo{faire un encart avec le lien de la mise en production. Ajouter aussi le gihub pour visualiser le code source}

			\todo{écire un environnement pour afficher le code inclut dans le projet}
	% --------------------------------------------------------------------------------
	% 	CREATION DE LA BASE DE DONNÉES
	% --------------------------------------------------------------------------------	
	\chapter{Création de la base de données}

		\section{Choix de la base de données}
			
		\section{Modèle Entité-Association}


	% --------------------------------------------------------------------------------
	% 	EXEMPLE D'UTILISATION DU SITE WEB
	% --------------------------------------------------------------------------------	
	\chapter{Utilisation du site web}

		\section{la page d'accueil}

		\section{Presentation du menu}

		\section{Création et connexion à son compte utilisateur}

			\begin{itemize}
				\item on utilise la table "users" de la base de données
				\item on décide de construit une unique page pour faire l'inscription/connexion
				avec du JS pour passer de l'un à l'autre.
				\item Pour une meilleure visualisation coté client, on utilise AJAX pour lancer nos requetes suites aux entrées clients. (fichier login-resonse.php)
			\end{itemize}

			\begin{rmq}
				Suite à l'utilisation d'AJAX pour passer nos requêtes, nous constatons que l'utilisation de la fonction "header('Location: index.php')" ne fonctionne plus. 

				Après recherches dans la documentation de PHP, nous découvrons que cela est dû au fait que cette fonction ne fonctionne que si aucun code html n'est encore affiché sur la page. Pour effectuer cette redirection, nous choisissons dorénavant d'utiliser la méthode Javascript "window.location.replace(url\_redirection);"  
			\end{rmq}

			\begin{rmq}
				Une petite subtilité de AJAX avec JQuery: l'utilisation "event.preventDefault();" pour éviter le rechargement de la page sur laquelle on se trouve.  Ce qui annule tout le travail d'Ajax et ce serait dommage quand même ;).
			\end{rmq}

		\section{Page de présentation d'un film}

			\subsection{Construction du système de notation des films}
				2 fonctionnalités : Afficher la note du film donné par les utilisateurs ET faire noter le film par l'utilisateur

				\subsubsection{Mise en place du système de notation pour l'utilisateur}

					On utilisera u système de notation allant de 1 à 5 pour noter les films et sera visualiser par un système d'étoiles.

					Les étoiles ici proviennent du framework "Font Awesome".

					\bigskip
					Mise en place de AJAX: deux formats de retour possibles: XML ou JSON. Comme le XML est généralement plus lourd et plus compliqué à manier, on préfère utiliser ici le JSON.

					Ici on a simplement, une méthode POST qui renvoie 3 choses: l'ID de l'utilisateur, l'ID du film à noter et la note. On pourrait presque simplement renvoyer du HTML ou passer directement les arguments dans la fenetre


			\subsection{Fonctions supplémentaires:}

				On propose à l'utilisateur des films en lien avec le film sélectionné:
				\begin{itemize}
					\item un film de même catégorie
					\item un film fait par le même réalisateur
					\item un autre film dans lequel a joué un des acteurs
				\end{itemize}

				POur cela on va utiliser des requetes SQL (via PDO de PHP) pour récupérer les résultats puis nous tirerons au hasard un des ses résultats pour l'afficher à l'aide d'une fonction incluse dans PHP : "array\_rand"



		\todo{Créer une annexe avec l'arborescence des fichiers du projet}
\end{document}



















