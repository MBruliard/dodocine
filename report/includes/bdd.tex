\chapter{Architecture du site web}

    Pour ce projet, nous avons essayé de tirer parti du dynamisme de PHP en créant de multiples fichiers annexes appelés à de multiples reprises dans différentes pages. Ces fichiers sont regroupés dans le dossier \f{elements}. On y retrouve notamment des parties de codes réutilisées pour chaque page de notre site

    \begin{itemize}
        \item \f{elements/head.php} contient la balise 'head' que nous rappelons pour chaque page de notre site. Nous lui avons ajouté deux variables , \f{\$title\_page} qui permet de changer le titre de la page au besoin ainsi que \f{\$css\_addon}, qui prend une chaine de caractères pour ajouter des liens vers des fichiers css supplémentaires.
        \item \f{elements/footer.php} contient le footer, répété dans chacune de nos pages.
        \item \f{elements/js\_files.php} appelle l'ensemble des scripts Javascript/JQuery dont nous avons besoin. De le même façon que pour \f{elements/head.php}, nous avons construit une variable \f{\$js\_addon} pour ajouter des scripts supplémentaires dans certaines pages.
        \item \f{elements/navigation.php} la barre de navigation
    \end{itemize}

    
    \section{Mise en place de la base de données}

        Nous avons fait le choix de créer des scripts PHP contenant l'ensemble des fonctions qui interagiront avec la base de données via le système PDO de PHP. Ces scripts sont regroupés dans le dossier \f{controller}:

        \begin{itemize}
            \item \f{controller/authentification.php} pour les fonctions relatives aux utilisateurs
            \item \f{controller/datafilms.php} pour les fonctions relatives aux films et à la notation de ceux-ci
            \item \f{controller/dataindividus.php} pour les fonctions permettant d'obtenir des informations sur les acteurs/réalisateurs contenus de la base de données.
            \item \f{controller/dataforum.php} pour la gestion du forum.
        \end{itemize}
        

        \bigskip
        Pour ce projet, nous avons fait le choix d'utiliser le système de gestion de base de données nommé SQLite et non pas MySQL, principalement afin de pouvoir avoir un hébergement gratuit de notre site pour sa présentation. Cependant, si SQLite est adaptée pour un projet universitaire, il engendre des ralentissements de navigation lorsque la taille de la base de données ou le traffic de visites dépassent un certain seuil.

        \medskip
        On remarque, que dans l'écriture de requêtes SQL avec PDO, SQLite ferme automatique la connexion à la base de données contrairement à MySQL.
