\chapter*{Introduction}
\addchaptertable{Introduction}
        
    \section*{Présentation du projet}
    \addsectiontable{Présentation du projet}

        De nos jours, l'arrivée en masse des plateformes de "streaming" sur Internet a permi de populariser l'accès aux films de tous types et de toutes générations. Afin d'aiguiller et d'aider les cinéphiles en herbes que nous sommes, nombre de sites nous aident aujourd'hui à découvrir des films qui correspondent à nos préférences. Parmi ces sites, nous retrouvons notamment \url[le site Allociné]{https://allocine.fr} qui présente des films ainsi que les acteurs et les réalisateurs y ayant participé.

        \medskip
        Dans le cadre du module {\bfseries \textit Programmation Web} de la licence d'Informatique de l'Universite Sorbonne Paris Nord, nous nous sommes penchés sur le fonctionnement de ce type de site web et avons choisit de construire un "mini-clone" du site d'Allociné, que nous appellerons en cette période de pandémie {\bfseries DodoCiné} afin de parfaire nos connaissances sur la programmation web avec les langages de programmation PHP\footnote{nous utiliserons ici la version de PHP7} et Ajax coté serveur et les langages HTML5/CSS3/Javascript pour le côté client.
    
        \vfill
        \boiteinfo{}{
            Il est possible de visualiser notre site {\bfseries DodoCiné} directement en cliquant sur ce lien: \url{http://dodocine.infinityfreeapp.com/}. 

            \medskip
            Cependant, notre hébergement étant gratuit, il est à noter que la protection de la base de données n'est pas assuré. En conséquence, nous ne vous conseillons pas d'utiliser votre véritable adresse email si vous souhaiter créer un compte
        }

        \vfill
        \boiteinfo{accès au github}{
            Vous pouvez retrouver l'ensemble de nos fichiers sources sdans notre repository Git: \url{https://github.com/MBruliard/dodocine}
        }