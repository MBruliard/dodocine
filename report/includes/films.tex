\chapter{Présentation des films} % nom du chapitre


    \section{Présentation des films disponibles sur le site} 

        Nous proposons dans l'onglet "Cinéma" de notre menu de navigation, la liste complète des films disponibles sur notre site. De cette façon, le visiteur peut découvrir au gré de ses envies de nouveaux films.

        \begin{figure}[!ht]
            \centering
            \includegraphics[width=16cm]{img/list-films.png}
            \caption{Page Cinéma de notre site}
        \end{figure}

        \medskip
        Les films sont ici regroupés par catégorie et présentés sous la forme d'un "slider carousel". Nous avons notamment trouvé un framework {\bfseries \href{Flickity}{https://flickity.metafizzy.co} } open-source (licence GPLv3) pour les projets personnels et universitaires que nous avons choisi ici d'utiliser. Ce framework est "responsive" et donc s'adapte parfaitement à notre site.

        \medskip
        L'affichage des "slider carousel" est géré automatiquement: pour chaque catégorie, nous récupérons via une requête PDO l'ensemble des films correspondant à la catégorie dans un tableau (array en PHP). Nous appelons ensuite le script PHP \f{elements/films-slider.php} auquel on applique le tableau de films. 

        \bigskip
        \section{Page de présentation d'un film}


            En cliquant sur un film de notre liste, le visiteur peut afficher une fiche descriptive du film sur laquelle sont renseignées des informations telles que l'année de sortie du film, son réalisateur, les acteurs principaux y ayant participés, etc.

            \medskip
            En plus de cela,  on propose à l'utilisateur des films en lien avec le film sélectionné:
                \begin{itemize}
                    \item un film de même catégorie
                    \item un film fait par le même réalisateur
                    \item un autre film dans lequel a joué un des acteurs
                \end{itemize}

                Pour cela on va utiliser des requetes SQL (via PDO de PHP) pour récupérer les résultats puis nous tirerons au hasard un des ses résultats pour l'afficher à l'aide d'une fonction incluse dans PHP : "array\_rand". L'algorithme se révèle en réalité très semblable à celui utilisé pour l'affichage aléatoire d'un acteur et d'un film sur la page d'accueil du site.

            \medskip

            Enfin, nous proposons aux utilisateurs ayant un compte sur notre site plusieurs fonctionnalités supplémentaires:
            \begin{itemize}
                \item noter un film
                \item ajouter un film à ses favoris\footnote{c.f. section sur l'espace membre}
                \item écrire dans le forum
            \end{itemize}


            \subsection{Construction du système de notation des films}

                Pour ce faire, nous utilisons un système de notation très à la mode en ce moment: la notation par étoiles (5 étoiles maximum). Les films sont alors notés sur 5.

                \boiteinfo{Font Awesome}{
                    Les étoiles, ainsi que la plupart des icônes présentes sur les boutons de notre site proviennent du framework \erd{Font Awesome}.
                }

                La note du film est la moyenne de toutes les notes données par les utilisateurs du site. De plus, un utilisateur connecté peut noter/modifier sa note sur un film en cliquant simplement sur le bouton "Noter ce film". Une fenêtre modale (une fenêtre modale intégrée à Boostrap) apparait alors pour que l'utilisateur entre sa note.

                \begin{figure}
                    \centering
                    \includegraphics[width=9cm]{img/noter-film.png}
                    \caption{Fenêtre modale permettant de noter un film}
                \end{figure}

                \bigskip
                L'enregistrement de la note de l'utilisateur, et la mise à jour de la note globale du film se fait ensuite via AJAX.\footnote{Nous avons fait le choix de n'utiliser que le format JSON pour le transfert de données via AJAX car celui-ci est moins lourd et plus simple à manier}. On utilise une méthode POST qui renvoie 3 choses: l'ID de l'utilisateur, l'ID du film à noter et la note.

                \medskip
                Lorsque la fenêtre modale se referme, la fonction AJAX met directement à jour les informations sur la note.

            
            \subsection{Gestion du forum relatif au film}

                \begin{figure}[!ht]
                    \centering
                    \includegraphics[width=16cm]{img/espace-commentaires.png}
                    \caption{Exemple d'espace commentaire (ici pour le film {\itshape Lucy})}
                \end{figure}

                Dans ce projet, on considère que les utilisateurs connectés peuvent écrire dans le forum du film pour discuter entre eux. Si un visteur du site n'a pas de compte, il peut lire les messages postés mais ne peut pas écrire ou répondre à des messages.

                \medskip
                Un utilisateur connecté peut:
                \begin{itemize}
                    \item écrire un message dans le forum du film
                    \item supprimer un message qu'il a posté auparavant
                \end{itemize}


                

                \bigskip
                \boiteinfo{Liste des fichiers utilisés}{
                    \begin{itemize}
                        \item \f{elements/forum-message.php} : pour écrire un message
                        \item \f{elements/forum".php} pour la gestion du forum. Il sera appelé pour chaque film
                        \item \f{static/css/forum.css} pour le style
                        \item \f{static/js/forum.js} pour la gestion des evenements
                        \item \f{controller/dataforum.php} pour les fônctions faisant appel a la base de données
                        \item \f{forum-response.php} pour les appels ajax
                    \end{itemize}
                }

                

                Finalement le fichier \f{elements/forum-message.php} n'est pas utilisé dans la construction du forum car nous ne sommes pas parvenu à faire l'appel succesif. A la place, nous écrivons directement la partie HTML à l'aide de l'appel de \f{forum-response.php} via AJAX. Ce n'est probablement pas la méthode la plus simple, ou la plus efficiente mais il s'agit de la seule méthode que nous avons trouvée...

                \medskip
                En plus d'afficher le message, son auteur et la date à laquelle il a été posté, nous créons un bouton permettant à un utilisateur de supprimé son propre post. Ce bouton n'est bien sûr visible que par lui lorsqu'il est connecté à son compte.