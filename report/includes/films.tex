\chapter{Présentation des films}


\section{Présentation des films disponibles sur le site}

            On propose sur cette page de retrouver tous les films disponibles sur la plateforme. L'utilisateur peut ainsi rechercher le film qu'il désire, en tapant directement le nom du film dans la barre de recherche, ou plus simplement en parcourant les différentes listes présentées.

            Chaque liste de films (présentée sous la forme d'un "slider carousel"\footnote{Pour la mise en place du "slider carousel", nous nous sommes largement inspirées du code source présenté par {\bfseries WebShalla} dans \href{https://www.youtube.com/watch?v=Gi4CTYOs7J4}{cette vidéo} }) correspond à une catégorie de films.

            Ou alors là je viens de trouver un framework qui s'appelle {\bfseries \href{Flickity}{https://flickity.metafizzy.co} } qui à l'air vraiment top alors on va essayer. Ce framework est en "open-source" (licence GPLv3) pour les projets personnels et universitaires ce qui nous convient parfaitement.


        \section{Page de présentation d'un film}

            \subsection{Construction du système de notation des films}
                2 fonctionnalités : Afficher la note du film donné par les utilisateurs ET faire noter le film par l'utilisateur

                \subsubsection{Mise en place du système de notation pour l'utilisateur}

                    On utilisera u système de notation allant de 1 à 5 pour noter les films et sera visualiser par un système d'étoiles.

                    Les étoiles ici proviennent du framework "Font Awesome".

                    \bigskip
                    Mise en place de AJAX: deux formats de retour possibles: XML ou JSON. Comme le XML est généralement plus lourd et plus compliqué à manier, on préfère utiliser ici le JSON.

                    Ici on a simplement, une méthode POST qui renvoie 3 choses: l'ID de l'utilisateur, l'ID du film à noter et la note. On pourrait presque simplement renvoyer du HTML ou passer directement les arguments dans la fenetre


            \subsection{Fonctions supplémentaires:}

                On propose à l'utilisateur des films en lien avec le film sélectionné:
                \begin{itemize}
                    \item un film de même catégorie
                    \item un film fait par le même réalisateur
                    \item un autre film dans lequel a joué un des acteurs
                \end{itemize}

                POur cela on va utiliser des requetes SQL (via PDO de PHP) pour récupérer les résultats puis nous tirerons au hasard un des ses résultats pour l'afficher à l'aide d'une fonction incluse dans PHP : "array\_rand"

\subsection{Gestion du forum relatif au film}

                Dans ce projet, on considère que les utilisateurs connectés peuvent écrire dans le forum du film pour discuter entre eux. Si un visteur du site n'a pas de compte, il peut lire les messages postés mais ne peut pas écrire ou répondre à des messages.

                \medskip
                Un utilisateur connecté peut:
                \begin{itemize}
                    \item écrire un message dans le forum du film
                    \item répondre à un message déjà posté (via le petit bouton) \todo{faire une capture d'écran pour montrer}
                    \item supprimer un message qu'il a posté auparavant
                \end{itemize}


                \bigskip
                Idées à mettre en place si on a le temps:
                \begin{itemize}
                    \item créer des filtres pour éviter les insultes dans les messages
                    \item créer des tables en cas de trop nombreux messages pour ne pas avoir des pages trop longues 
                \end{itemize}

                \bigskip
                Fichiers :
                \begin{itemize}
                    \item "elements/forum-message.php" : pour écrire un message
                    \item "elements/forum".php pour la gestion du forum. Il sera appelé pour chaque film
                    \item "static/css/forum.css" pour le style
                    \item "static/js/forum.js" pour la gestion des evenements
                    \item "controller/dataforum.php" pour les fônctions faisant appel a la base de données
                    \item "forum-response.php" pour les appels ajax
                \end{itemize}

                Finalement le fichier "elements/forum-message.php" n'est pas utilisé dans la construction du forum... A la place on passe directement cette partie du code dans le fichier "forum-response.php" pour renvoyer du code html. Ce n'est pas très joli au niveau du code mais ça marche et c'est la seule méthode que j'ai trouvé...